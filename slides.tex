\documentclass{beamer}
\beamertemplatenavigationsymbolsempty
\usepackage{graphicx}
\usepackage{listings}
\lstset{
    breaklines=true,
    numbers=left,
    numberstyle=\scriptsize,
    frame=leftline,
    basicstyle=\ttfamily}

% TODO
%  * use \begin{frame}[fragile] with code examples


\begin{document}
\begin{frame}

    {\LARGE Uvod do Bitcoinu}\\
    \vspace{7mm}
    {\Large Ondrej Sika \lstinline|<ondrej@ondrejsika.com>|}\\
    \vspace{7mm}
    {\large Slush Pool (\url{mining.bitcoin.cz})}\\
    \vspace{7mm}
    26. 8. 2015, Pilsen, Czech Republic\\

\end{frame}

\begin{frame}

    {\LARGE Co je Bitcoin}\\
    \vspace{5mm}
    {\LARGE Jak Bitcoin funguje}\\
    \vspace{5mm}
    {\LARGE Bitcoinove penezenky}\\
    \vspace{5mm}
    {\LARGE Kde koupim Bitcoiny}\\
    \vspace{5mm}
    {\LARGE Kde utratim Bitcoiny}\\
    \vspace{5mm}
    {\LARGE Jak prijimat Bitcoiny}\\

\end{frame}

\begin{frame}

    {\LARGE Co je Bitcoin}\\

    \vspace{5mm}

    - peer to peer digital cash - neni nutny prostrednik pro transakce\\
    - distribuovany - ulozen v mnoha kopiich na celem svete\\
    - decentrializovany - nema centralni autoritu\\
    - anonymni\\

\end{frame}

\begin{frame}

    {\LARGE Jak Bitcoin funguje}\\

    \vspace{5mm}

    - zalozeno na cryptografickych zakladech\\
    - transakce je podepsany prikaz k prevodu penez konkretni penezenkou (privatni klic) na jinou penezenku (verejny klic)\\
    - transakce jsou ukladany do blockchainu, posloupnost bloku transakci bez nutnosti centralni autority\\
    - mining - generovani novych coinu a bloku transakci\\

\end{frame}

\begin{frame}

    {\LARGE Bitcoinove penezenky}\\

    \vspace{5mm}

    - hardwarove - Trezor\\
    - PC - Bitcoin qt, Electrum\\
    - mobilni - Mycelium, Coinbase\\
    - web - Coinbase\\
    - paperwallets\\

\end{frame}

\begin{frame}

    {\LARGE Trezor}\\

    \vspace{5mm}

    - Nejbezpecejsi zpusob ulozeni Bitcoinu\\
    - Online klient \url{https://MyTrezor.com}\\
    - Podpora v Androin Myceliu\\
    - Vice na \url{http://bitcointrezor.com}\\
    - Produkt nasi spolecnosti (SatoshiLabs)\\

\end{frame}

\begin{frame}

    {\LARGE Dekuji za poroznost \& Otazky \& Diskuze}\\

    \vspace{1cm}

    \texttt{ondrej@ondrejsika.com}\\
    \url{http://ondrejsika.com}\\
    \texttt{@ondrejsika}\\

    \vspace{1cm}

    Sources:\\
    \url{http://url.os1.cz/uvod-do-bitcoinu/}
\end{frame}

\end{document}

